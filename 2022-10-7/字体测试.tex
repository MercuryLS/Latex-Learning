%导言区
\documentclass[10pt]{ctexart}
\newcommand{\myfont}{\textit{\textbf{\textsf{Fancy Text}}}}

%正文区
\begin{document}
%字体族设置(罗马,无衬线,打印机)
    \textrm{Roman Family}   \textsf{Sans Serif Family}   \texttt{Typewriter Family}
    {\rmfamily Romen Family}    {\sffamily Sans Serif Family}   {\ttfamily Typewriter Family}   %大括号表示字体的作用范围
    {\sffamily who are you?}
    
    {\ttfamily who are you?}
    
    who are you?
    
    who are you?

%字体系列设置(粗细,宽度)
    \textmd{Medium Series} \textbf{Boldface Series}
    {\mdseries Medium Series} {\bfseries Boldface Series}

%字体形状设置(直立,斜体,伪斜体,小型大写)
    \textup{Upright Shape} \textit{Italic Shape}
    \textsl{Slanted Shape} \textsc{Small Caps Shape}

    {\upshape Upright Shape} {\itshape Italic Shape}
    {\slshape Slanted Shape} {\scshape Small Caps Shape}

%中文字体
    {\songti 宋体} \quad    %这是空格,中文空格要用\quad,直接在中文的源敲是没有空格的,英文可以直接空格
    {\heiti 黑体}  \quad
    {\fangsong 仿宋} \quad
    {\kaishu 楷书}
    %{\textbf 粗体}
    %{\textit 斜体}

%字体大小
    {\tiny       Hello}\\%\\表示另起一行
    {\scriptsize Hello}\\
    {\footnotesize Helo}\\
    {\small Hello}\\
    {\normalsize Hello}\\
    {\large Hello}\\
    {\Large Hello}\\
    {\LARGE Hello}\\
    {\huge Hello}\\
    {\Huge Hello}\\

%中文字号设置命令
    \zihao{5}你好!\\
    \myfont
    \end{document}
