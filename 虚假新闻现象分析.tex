\documentclass{ctexart}
\usepackage{amsmath,amsthm,amssymb,graphicx}
\usepackage{fancyhdr}% 设hi页面样式
\pagestyle{fancy}
\renewcommand{\headrulewidth}{0pt}% 去掉fancy中的眉线
\fancyhf{}
\cfoot{\thepage}
\CTEXsetup[format={\normalfont\normalsize}]{subsubsection} % 让subsubsection中的文字不被加黑
\linespread{1.5} % 1.5倍行距
\setCJKmainfont[SizeFeatures={Size=12},AutoFakeBold=2.5,AutoFakeSlant=0.2]{SimSun} % 全文默认小四号宋体


\title{\heiti \textbf{\zihao{4}虚假新闻现象分析}}
\author{\heiti \zihao{-4}2022302021175\quad \heiti \zihao{-4}物理科学与技术学院\quad \heiti \zihao{-4}刘湛}
\date{}
\begin{document}
\maketitle
\par
\section*{}
\noindent
{\heiti \zihao{-4}摘要:}虚假新闻现象是近年来互联网的流行话题之一,经常被提及和讨论。随着媒体工具的发展和互联网的广泛使用,虚假信息已成为一种全球范围内的现象。各种成因如信息泛滥、商业利益等都使得虚假新闻现象的影响越来越大,已经渗透到社会的多个方面。本文将介绍虚假信息的性质和类型,分析虚假新闻的危害以及应对虚假新闻现象的方法。

\noindent
{\heiti \zihao{-4}关键词:} 虚假信息;虚假新闻;应对虚假新闻
\\ \hspace*{\fill} \\
\indent
虚假新闻现象是一个当前备受关注的问题。虚假新闻指的是被有意地扭曲、变形、删除或者添加信息的新闻,其目的是欺骗受众,获取流量。由于具有制造便捷、传播广泛的性质,虚假新闻实质上已成为严重的公共安全问题。虚假新闻的出现和流行固然与现代媒体技术和互联网应用的发展直接相关,但更体现了社会、经济、政治等各方面因素的高度融合。随着时代的发展,虚假新闻现象越来越普遍,不仅对公众造成了诸多的负面影响,而且也威胁到了现代社会的可持续发展。因此,我们有必要分析分析虚假新闻。
\subsection*{\textbf{一、虚假新闻的定义及特点}}
虚假新闻指的并不是指新闻本身不够真实,而是指其报道的事实并不属实的新闻。虚假新闻的主要特点是缺乏实际的信息来源和真实的证据支持。而虚假新闻的出现有时可能会产生巨大的影响,影响甚至会扩散到整个社会。现代社会的媒介对于虚假新闻的传播起到了推波助澜的作用,这些媒介往往包括社交媒体,网络论坛甚至是电视新闻。其中,社交媒体,如微博、知乎和小红书等,在虚假新闻传播中扮演着越来越重要的角色。同时,虚假新闻越来越复杂和高度工程化,因此很难在短时间内识别真伪,这使得虚假新闻极具欺骗性。
\subsection*{\textbf{二、虚假新闻的产生原因}}
分析虚假新闻产生的原因,对于我们识别、规避虚假新闻大有裨益。不难发现,虚假新闻的背后往往有以下几个推手:
\subsubsection*{\quad \quad(一)商业化驱动}
在现代社会,新闻媒体往往以追求商业利益为主要目标,而为了吸引受众的注意力和提高点击率,有时会不惜夸大事实、捏造细节、甚至编造整个新闻事件来获取更大的收益,从而严重损害了新闻的真实性和公信力。
\subsubsection*{\quad \quad (二)政治利用}
虚假新闻在政治宣传中也是常见的手段之一。各种政治势力或团体为了获取更多的民意支持或拉拢更多的人才,经常会采用夸大和歪曲事实、混淆是非黑白、抹黑攻击对手等手段。虚假新闻不仅会影响公众的政治判断和参与,而且也会干扰政策的制定和决策的执行。在媒体发达的国家,这种情况会更加严重。2021年11月3日,也即美国总统大选投票日当天,一些社交媒体上出现了一些声称“民主党在作弊”的虚假新闻。 例如,有人发布了一张照片,显示一名穿着“拜登-哈里斯”T恤衫的男子在一台投票机旁边站着,暗示他在操纵投票结果。 然而,经过媒体调查,该照片是在10月24日拍摄的,而且该男子只是在帮助一位老年选民使用投票机,并没有任何不正当行为。 这样的虚假新闻很可能是由一些支持特朗普的政治势力或个人制造或转发的,目的是为了影响选举结果和公众信心。
\subsubsection*{\quad \quad (三)社交媒体的传播}
由于社交媒体的普及、便捷以及隐匿性,个人可以很容易地创建虚假的网络身份,并且可以将虚假的消息传递给广大的网络用户而逃脱追责。这些虚假的消息如果被更多的用户分享、转发、评论从而形成雪崩式的链式反应是,会很容易转化成虚假新闻,并在网络空间形成巨大的影响力。2022年1月25日,抖音等短视频平台流传有“大雪天邯郸武安交警查酒驾致使车辆滑行数米,并造成人员受伤”的视频。该视频引发了网友的强烈关注和愤慨,有人质疑交警的执法方式和责任心。然而,经过官方核实,该视频是由两段不同时间、不同地点的视频拼接而成的,完全是虚假新闻。该视频的制作者和传播者很可能是为了获取更多的流量和收益,而不顾真相和社会影响,制造了这样的虚假新闻。自媒体的运营者往往会为了博取流量采取使用虚假新闻的方式,这严重危害了媒体的生态。
\subsection*{\textbf{三、虚假新闻的形式与类型}}
从虚假新闻的产生方式来看,我们可以将虚假新闻分成以下形式:
\subsubsection*{\quad \quad (一)彻头彻尾的虚假新闻}
现为将根本不存在的事件制造为新闻,如“中国每年有220万青少年死于室内污染”等。
\subsubsection*{\quad \quad (二)捕风捉影的虚假新闻}
表现为将若有若无、关系不大的事件牵强附会为新闻,如“女孩跟男友回农村过年,见到第一顿饭想分手了”等。
\subsubsection*{\quad \quad (三)渲染造作的虚假新闻}
表现为将小事扩大,肆意渲染,如“女员工每日排队吻老板”等。
\subsubsection*{\quad \quad (四)本末倒置的虚假新闻}
表现为故意曲解客观事实,蒙蔽事情真相,如“范冰冰母女共侍大佬”等。
\\ \hspace*{\fill} \\
\indent
从虚假新闻的内容来看,我们可以将虚假新闻分成以下类型:
\subsubsection*{\quad \quad (一)政治类虚假新闻}
指那些涉及政治人物、政治事件、政治立场等方面的虚假新闻,往往是由一些政治势力或个人出于政治目的而制造或传播的,目的是为了影响选举结果、公众信心、国际关系等。如“叙利亚诗人阿多尼斯获诺贝尔奖”等。
\subsubsection*{\quad \quad (二)社会类虚假新闻}
指那些涉及社会敏感问题、社会热点话题、社会公众关切等方面的虚假新闻,往往是由一些媒体或个人出于商业利益或社会影响而制造或传播的,目的是为了获取更多的流量、收益、关注等。如“江西九江发生6.9级地震”等。
\subsubsection*{\quad \quad (三)娱乐类虚假新闻}
指那些涉及娱乐明星、娱乐事件、娱乐话题等方面的虚假新闻,往往是由一些媒体或个人出于猎奇心理或娱乐效果而制造或传播的,目的是为了满足受众的好奇心、审美需求、情感寄托等。如“张杰谢娜离婚”等。
\subsubsection*{\quad \quad (四)正能量类虚假新闻}
指那些涉及正面价值观、正面道德观、正面情感观等方面的虚假新闻,往往是由一些媒体或个人出于弘扬正能量、倡导积极的价值观而制造或传播的,目的是为了激励受众、传递温暖、提升信心等。如“北大才女回乡创业送快递”等。
\subsection*{\textbf{四、虚假新闻的危害}}
虚假信息是所谓没有客观事实根据的“新闻”,必然会对大众产生巨大的负面影响,从作用对象来看,主要会产生以下影响:
\subsubsection*{\quad \quad(一)侵害受众的知情权和监督权}
受众有权获取真实、准确、完整的信息,参与公共事务和发挥舆论监督作用,但虚假新闻为受众提供了错误、夸张、片面的信息,误导受众,干扰公共决策,损害公共利益。虚假新闻可能对特定人群使用“筛选性传播”(selective sharing)策略,即有意把虚假新闻传递给特定的受众,操纵受众和信息传播渠道。这类做法需要准确把握用户的兴趣、偏好和地理位置等因素,以便在传播媒介中定制化信息给用户,让他们更容易形成偏见,达到特定的传播目的。
\subsubsection*{\quad \quad(二)损害媒体和记者的公信力和职业道德}
对于媒体人来说,虚假信息的出现意味着信息渠道受到操控,尊重事实的原则被违背,新闻的可信度遭到降低。如果新闻机构不能向公众证明其新闻的真实性与可靠性,那么媒体与记者的职业道德便会受到怀疑。
\subsubsection*{\quad \quad(三)影响国家安全和社会稳定}
虚假新闻往往涉及政治、经济、军事、外交等敏感领域,可能被敌对势力利用,对国家形象、国际关系、民族团结等造成负面影响。虚假新闻也可能引发民众恐慌、不满、暴力等社会问题,扰乱社会秩序,危害社会和谐。
\subsection*{\textbf{五、防范虚假新闻的措施}}
虚假新闻对于社会有着巨大的危害,因此,我们必须采取措施来防范虚假新闻为祸社会。从责任主体来看,我们可以采取以下措施:
\subsubsection*{\quad \quad(一)新闻机构应当加强自我审查}
闻媒体可以对舆论产生巨大影响,正因如此,新闻机构应当加强自我审查来减轻虚假乃至新闻的影响。新闻机构必须对所有的报道进行再次核实,并且要正确引用来源和数据,如果新闻机构不确定某些信息的真实度,可以选择不进行报道,从而保护自己的信誉并制止虚假新闻的传播。此外,还应当加强新闻工作者的职业道德教育,树立正确的世界观、人生观、价值观,恪守新闻真实性原则,严格遵守新闻从业人员职业道德准则.
\subsubsection*{\quad \quad(二)采取法律、行政、民间力量处置虚假新闻}
针对虚假新闻问题,国家应严格制定相关法律,明确新闻媒体的权利和义务,并依据确证新闻源并确认信息真实性。对于刻意传播虚假新闻并产生不良影响的应当依据法律进行追究。同时,政府的宣传部门应当建立健全事实核查机制,积极开通辟谣平台,多方采访信源,加强事实核实,对虚假新闻进行及时揭露和纠正。对于热心群众,可以发动多元社会力量,促进共同参与网络内容生态治理,加强新闻法规建设,强化行业自律,对虚假新闻制造者和传播者进行谴责。
\subsubsection*{\quad \quad(三)加强公众教育}
在自媒体时代,网民在信息的传播中扮演着重要的角色。因此对于网络媒体的用户,教育就显得尤为重要。公众需要了解虚假新闻的危害,理解新闻媒体的新闻生产流程和标准。同时,还要增强公众的媒介素养和辨别能力,培养理性、客观、审慎的网络使用习惯,不轻信、不转发、不传播未经核实的信息。
\subsection*{\textbf{六、结论}}
虚假新闻是当今社会的一种严重的信息污染现象,它不仅侵害了公众的知情权和监督权,损害了媒体和记者的公信力和职业道德,而且影响了国家安全和社会稳定。因此,我们必须从多方面采取措施来防范虚假新闻的危害,只有这样,我们才能建立一个真实、公正、健康的新闻环境,促进社会的进步和与人类的发展。
\end{document}